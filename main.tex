\documentclass[conference]{IEEEtran}
% The preceding line is only needed to identify funding in the first footnote. If that is unneeded, please comment it out.
\usepackage{cite}
\usepackage{amsmath,amssymb,amsfonts}
\usepackage{algorithmic}
\usepackage[inline]{enumitem}
\usepackage{graphicx}
\usepackage{textcomp}
\usepackage{xcolor}
\def\BibTeX{{\rm B\kern-.05em{\sc i\kern-.025em b}\kern-.08em
    T\kern-.1667em\lower.7ex\hbox{E}\kern-.125emX}}
\begin{document}

\title{Research directions for Aggregate Computing with Machine Learning}

\author{\IEEEauthorblockN{PhD Student Gianluca Aguzzi}
\IEEEauthorblockA{\textit{Alma Mater Studiorum - Università di Bologna} \\
Cesena, Italy \\
gianluca.aguzzi@unibo.it}
% \and
% \IEEEauthorblockN{Supervisor Mirko Viroli}
% \IEEEauthorblockA{\textit{Alma Mater Studiorum - Università di Bologna} \\
% Cesena, Italy \\
% mirko.viroli@unibo.it}
}

\maketitle
\begin{abstract}
    Collective adaptive systems are challenging from the engineering perspective. Different techniques aim at taming those challenges, either using declarative or black-block approaches (e.g. Machine Learning, Evolutionary Algorithms, etc.).
    Among the many declarative approaches, Aggregate Computing is a novel technique by which developers can express collective system behaviour from a global perspective using a compositional and functional programming technique.
    Over the years, this approach has been applied in different scenarios, ranging from smart cities to a crowd of people. Despite its promising capabilities, sometimes describing aggregate behaviour is tricky and tedious, so we aim at merging Aggregate Computing with black-box techniques to understand the implications and the benefits of this combination.
\end{abstract}
\begin{IEEEkeywords}
Field Coordination, Aggregate Computing, Machine Learning, Multi-Agent Reinforcement Learning
\end{IEEEkeywords}

\section{Motivation and challenges}
Humans, in the last century, have filled the earth with computational entities. 
%
We find things able to compute in smartphones, fridges, and watches. % improve this period
Nowadays, these devices are not isolated; rather, they form a hoard of inter-communicating entities capable of achieving collective tasks. 
These systems exhibit properties usually observed in complex ones, such as large scale, 
decentralised control, and the global behaviour that emerges from local interactions~\cite{DBLP:conf/huc/Ferscha15}. 
Swarm of UAVs, a crowd of people, and smart cities are all instances of this kind of systems, usually called Collective Adaptive Systems.

In literature, different techniques aim at taming this complexity at the engineering level.
Among the many, two solutions that use a top-down approach in expressing global behaviour are TOTA~\cite{DBLP:journals/tosem/MameiZ09} and Aggregate Computing (AC)~\cite{DBLP:journals/computer/BealPV15} (both classified under the umbrella of Field-Based Coordination~\cite{DBLP:books/daglib/0015276}).
The latter represents a novel approach in which developers can express the collective behaviour declaratively by
a functional manipulation of a distributed data structure called a \emph{computational field}.

AC is backed by field calculus, a minimal set of instructions necessary to express whatever spatiotemporal computation 
through the manipulation of computational fields~\cite{DBLP:conf/coordination/AudritoBDV18}. 
On top of that, different building blocks have been built to facilitate the high-level libraries definition.
Furthermore, crafting a specific blocks category, is possible to verify relevant properties such as self-stabilization~\cite{DBLP:conf/coordination/ViroliD14} and eventual consistency~\cite{DBLP:conf/saso/BealVPD16}.
Practically though, building these basic blocks is not as easy as it sounds. 
Moreover, it is challenging to guarantee certain quality levels under different network conditions (e.g., different typologies, high node mobility, etc.)

Outward of top-down and declarative approaches, in the context of Multi-Agent Systems, other solutions leverage evolutionary computing~\cite{DBLP:journals/swarm/BrambillaFBD13} and machine learning (in particular Reinforcement Learning (RL)~\cite{DBLP:journals/access/NaeemRC20} and Multi-Agent RL~\cite{DBLP:journals/tcyb/NguyenNN20}) to design distributed programs.
Exploiting these techniques conducts to near-optimum solutions for specific and complex tasks such as multiplayer games~\cite{DBLP:journals/nature/VinyalsBCMDCCPE19} and traffic control~\cite{DBLP:journals/aes/JinMK17}.
Unfortunately, though, there can be many different problems to overcome with those solutions, such as
the difficulty in scaling up application complexity, the reusability of the behaviour found in other circumstances, and the "black-box" nature. 
%1
Definitely, the AC model is challenging in terms of usual learning problems. Indeed:
\begin{enumerate*}[label=(\roman*)]
\item there is no population size guarantee,
\item the neighbourhood is potentially highly dynamic,
\item the system behaviour is usually evaluated from the macro-level and not from the micro-level, and
\item each node has partial information about the system (in typical Dec-POMDP~\cite{DBLP:conf/uai/BernsteinZI00} settings).
\end{enumerate*}
\section{Contribution and objectives}
As far as we know, these two viewpoints are disjointed even if they speak a similar language. So our first intuition, and our key objective,
was to merge them to gain benefits from both approaches. This fusion creates declarative,
composable and self-explanatory language combined with online adaptivity and robustness towards environmental changes.
%
This has inevitably led to the following questions that guided us in the preliminary research phases: at what level of abstraction (middleware? Application API? Building blocks?) learning can be useful for AC? 
In state-of-the-art related techniques, how can they handle these extremely non-deterministic and complex systems?  
%
\section{Methodology and preliminary results}
%
As the first step, we consider related work in Multi-Agent Systems and Swarm Robotics because, in some cases, problems have similar settings.
To reach our objective, we iteratively 
\begin{enumerate*}[label=(\roman*)]
\item read works at the state-of-the-art,
\item take inspiration from one of them and craft an initial prototype, and
\item try to generalise the solution from complex problems.
\end{enumerate*}
AC currently is supported by different toolkits as ScaFi~\cite{DBLP:conf/ecoop/CasadeiV16} and Protelis~\cite{DBLP:conf/sac/PianiniVB15}, giving us fast prototype capabilities.
In this first part, we were more focused on practical results than foundational ones, hiding details of engineering hybrid systems like these.  

Currently, we are applying learning algorithms at the building blocks level; because they have an outstanding effect at the application level but
at the same time, the behaviours expressed are simple, so easily synthesised via learning.
Besides, the collective behaviour is cooperative and homogeneous, namely each agent performs the same local behavior~\cite{DBLP:journals/aamas/PanaitL05}.  %%improve settings
%
Our test application is the distance gradient from a source zone. %% TODO improve here.
It is an idiomatic pattern useful for Field-Based calculus for establishing different collective behaviour ranging from distributed sensing to information diffusion.
%Moreover, it is used to build other complex patterns like Self-organising coordination regions~\cite{DBLP:conf/coordination/CasadeiPVN19}.
%
Naive gradient implementations suffer from slow-rising, so different hand-crafted solutions have been built, such as Flex, BIS, and ULT~\cite{DBLP:conf/saso/AudritoCDV17}.
Our goal is to reach a similar (or even better) performance of the best hand-crafted solution (the ULT gradient).
%
In our workflow, we evaluate our result leveraging simulations, because the cost of setting up a large scale system.
In the AC literature, the Alchemist simulator is often choosed thanks to it's adaptbility of different application kind, ranging from crowd to smart cities.
% 
From related works, we observe a current trend in applying standard RL algorithms in distributed applications.
Q-Learning~\cite{DBLP:journals/ras/Krose95}, even if it is born for a single agent and a stationary environment, it is applied in different scenarios~\cite{DBLP:conf/mass/ShahK07}, also with some improvements (e.g. Hysteretic Q-Learning~\cite{DBLP:conf/iros/MatignonLF07} tries to manage the non-stationary problem nature).
%It is worth noting that no convergence theorem to a global optimal policy exists in Multi-Agent settings. 
%Usually, Nash Equilibrium~\cite{DBLP:conf/icml/HuW98} is proven to assess a good Multi-Agent learning algorithm, but unfortunately not always is associated with a global optimal policy~\cite{DBLP:conf/uai/PeshkinKMK00}.
%
At the moment of writing, none of the following novel methods brings us to good results. They are investigations that will hopefully yield results in the future and currently led us to understand limitations and opportunities.
%
As the first attempt, we try to encode our gradient problem in terms of a learning problem and then use Q-Learning (or also the Monte Carlo Learning~\cite{DBLP:conf/nips/Thrun99}) to solve it.
These experiments have shed light on some difficulties for us, which are: encoding local state giving neighbours values (a problem in finding a good representation), finding a right reward function that expresses the utility individuals
regarding the collective result, encode right actions, highly non-stationary behaviour. 
%
In particular, the gradient problem is hard to encode as a learning task due to the continuous state and action space.
So, we also consider using Neural Network to perform regression tasks using Backprograpagation algorithms to
improve the model performance. 
Even if it is a promising idea, currently, this method doesn't scale in complex problems: even with a global system observation, it is arduous to know the correct result at a certain point in time, so 
we cannot give the right baseline for our learning algorithms.
%
Finally, observing evolutionary robotics and avoiding loss function definition, we try some basic evolutionary methods, such as NEAT~\cite{DBLP:journals/ec/StanleyM02}.
The benefits of evolutionary algorithms are the limited constraints imposed. Indeed, the encoding is straightforward, and the fitness function is simpler to define than the reward function.
The problems encountered with this approach are the slowness and the difficulty in generalisation.
%
\section{Future work and research plan}
One of the problems encountered is the difficulty of encoding the state space in a fixed and dense dimension. Indeed, the overall graph structure is dynamic, so each node might perceive different neighbours at each time step.
In literature, these problems are referred to as \emph{node embedding}. Different techniques aim at learning the embedding from a specific graph. One of them is Graph Neural Network (GNN)~\cite{DBLP:journals/tnn/ScarselliGTHM09}.
Even if the configuration is quite different from our model, we can take inspiration from this branch of Neural Network to possibly improve our learning processes.

Deep RL, thanks to the famous work of Atari games~\cite{DBLP:journals/corr/HosuR16}, has gained interest in the last few years. Novel works~\cite{DBLP:journals/aamas/Hernandez-LealK19} investigate the application of deep neural networks model in Multi-Agents context
to solve well-known problems (non-stationary, partially observability, large state space, etc.). Currently, due to the simple nature of building blocks, we do not apply these techniques. But, if we want to apply learning processes even if
at application layers, these techniques might be necessary.

Most works in Multi-Agent Systems do not consider the possible enormous large scale of real applications, even if some exceptions exists~\cite{nguyen2018reinforcement}. With AC we write applications \emph{scale free}, so we need to use/create learning algorithms and models able to work
regardless of the system scale, that currently is not considered in depth.

Beyond building blocks learning, AC needs a platform upon which to execute programs. 
Here, designers need to define aspects that the paradigm does not include, 
such as evaluation frequency, program displacement, and trust. 
These themes are partially tackled in different works~\cite{DBLP:journals/scp/CasadeiAV18, DBLP:journals/fi/CasadeiPPVW20, DBLP:journals/corr/abs-2012-13806}, 
but online learning is usually not considered even if here could be crucial, 
since it would avoid complex parameters tuning and improving the system robustness.
Regarding the evaluation frequency, related works already applied RL. 
Indeed, following an ad-hoc crafted reward function, we might bring the system to reduce energy consumption even in our solution.

In conclusion, we are trying to define a research path that brings us in a complete integration between the AC and learning models. Even if it is quite opaque, the main
steps that we expect to follow are:
% research plan
\begin{itemize}
    \item 1st year: (4 months remained): concluding experiments with basic building blocks, create a clear test base to compare the standard application with our new approach. Deep learning models, Multi-agent RL, and GNN ideas might be integrated to reach good solutions.
    \item 2nd year: learning building blocks -- using the renewed experience -- that currently have problems in expressing with AC (such as the distributed leader election).
    Defining guidelines in merging these two approaches (from an engineering point of view). Creating a sort of hybrid programming methodology.
    \item 3rd year: extending learning tasks to the framework level, closing the reality gap, making the deployment in real application efficient. Understanding implications of this new way of programming, inserting learning ideas at the AC model level. 
\end{itemize}
\section*{Acknowledgement}
%% improve
Special thanks to my PhD Supervisor Viroli Mirko.
\bibliographystyle{ieeetr}
\bibliography{mybibfile}
\end{document}
